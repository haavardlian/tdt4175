\subsection{Eksamensweb}

Eksamensweb is accessible from the innsida web-page. It gives everything you need to know about 
exams. The main web-page contains two parts; first, a static part giving contact details, location, 
postal address and examination regulations (i.e. such information as arrival time, calculators and 
dictionaries allowed and so on). Then a dynamic part containing 4 sub-parts listed below:

{\small{\textbf{Registration}}}

In this sub-part you can check your exam dates, register for exams (you will be redirected to 
Studentweb to keep going over this task) and check about exam dates and locations.
\par
Also, you have information about the deadlines for registration for both semesters, about 
courses that have a "re-sit" or re-schedule exams and even if you need to register after the 
deadline (only special cases).
\par
Below the previous section, there is insights about canceling your registration for exam in 
both semesters. This will also redirect you on Studentweb and you have to do it before the deadline otherwise you'll have to take the exam.
\par
Final section is about special accommodations, this concerns you if you have health problem 
or a disability and then you can ask for special accommodations during your examinations. 
However, you will need to ask for it and there are deadlines for both semesters.

{\small{\textbf{Preparations}}}

This sub-part permits you to deal with the organization of the exam. You can access reading rooms and study hall locations in order to study beforehand.
\par
Then, you are given a link to courses information about date, time and room for you to know where and when to go and take your exam. If you lack specific information, you'll be redirected on Studentweb again for further details.
\par
Finally, you can have details on previous examinations (organization details too) and how to 
master your examinations, that is to say getting help and improve your exam-taking skills and 
how examiners assess your work

{\small{\textbf{During examination}}}

This sub-part deals with the very moment of the exam, it explains how the attendance works: 
you must arrive 10 minutes before the beginning of the examination, bring a photo 
identification and show this one to the invigilators before you can sign the attendance list.
\par
You have a link that shows you the permitted examination aids, same link as the on in the 
static part of the webpage.
\par
Then, some insights about how it is during the exam, if you are all of a sudden overwhelmed 
by some illness, if you ever try to cheat on an examination and some details on the presence 
of teaching staff during examinations.
\par
Last section helps you to find the examination room. Indeed, it gives you a link of an 
overview of the lecture halls and rooms, a map of NTNU's campuses and the map of 
Trondheim Spektrum.

{\small{\textbf{Examination results}}

Last sub-part of the dynamic part, it deals with the results. It gives you the examination result 
deadline, the explanation of grades and appeals and grade scale. Once the results are posted 
online, you can check them on Studentweb.
\par
You are also given a transcripts and diplomas/certificates details. Here you can order a 
transcript or diploma/certificate and you can also ask for a diploma supplement.
\par
Last section concerns the new examination, if you ever need to retake an exam in order to 
improve your grade, you have the information you need. The very last point is about the re-
sit examination.